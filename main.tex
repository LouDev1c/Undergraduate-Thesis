% !Mode:: "TeX:UTF-8"
% !TEX program  = xelatex
% !BIB program  = biber
\documentclass[AutoFakeBold,AutoFakeSlant,language=chinese,degree=bachelor]{sustechthesis}
% 1. AutoFakeBold 与 AutoFakeSlant 为伪粗与伪斜,如果本机上有相应粗体与斜体字体,请使用 xeCJK 宏包进行设置,例如:
%   \setCJKmainfont[
%     UprightFont = * Light,
%     BoldFont = * Bold,
%     ItalicFont = Kaiti SC,
%     BoldItalicFont = Kaiti SC Bold,
%   ]{Songti SC}
%
% 2. language=chinese 基于为 ctexart 文类提供的中文排版方案修改,如果使用英文进行论文创作,请使用 language=english 选项。
%
% 3. degree=bachelor 为 sustechthesis 文类提供的本科生毕业论文模板,其他可选项为 master 与 doctor,但是均未实现,如果您对此有兴趣,欢迎 PR。
%
% 4. sustechthesis.cls 文类主要参考自去年完成使命的 sustechthesis.tex,在这一年的时间,作者的 TeX 风格与常用宏包发生许多变化,因为之前的思想为仅提供必要的格式修改相关代码,所以转换为文类形式所进行的修改较少,而近期的风格与常用宏包均体现在以下的例子文件中。
%
% 5. 示例文件均放置于相应目录的 examples 文件夹下,构建自己论文时可暂时保留,用以检索接口与使用方法。
%
% 6. 英文目录需要居中可以使用:\renewcommand{\contentsname}{\centerline{Content}}
%
% 7. LaTeX 中公式编号括号样式及章节关联的方法:https://liam.page/2013/08/23/LaTeX-Formula-Number/

% !Mode:: "TeX:UTF-8"
% !TEX program  = xelatex

% 数学符号与环境
\usepackage{amsmath,amssymb,amsthm}
  \newcommand{\dd}{\mathrm{d}}
  \newcommand{\RR}{\mathbb{R}}
% 参考文献
\usepackage[style=gb7714-2015,gbpunctin=false]{biblatex}
  \addbibresource{ref.bib}
% 无意义文本
\usepackage{zhlipsum,lipsum}
% 列表环境设置
\usepackage{enumitem}
% 浮动题不越过 \section
\usepackage[section]{placeins}
% 超链接
\usepackage{hyperref}
% 图片,子图,浮动题设置
\usepackage{graphicx,subcaption,float}
% 抄录环境设置,更多有趣例子请命令行输入 `texdoc tcolorbox`
\usepackage{tcolorbox}
  \tcbuselibrary{xparse}
  \DeclareTotalTCBox{\verbbox}{ O{green} v !O{} }%
    {fontupper=\ttfamily,nobeforeafter,tcbox raise base,%
    arc=0pt,outer arc=0pt,top=0pt,bottom=0pt,left=0mm,%
    right=0mm,leftrule=0pt,rightrule=0pt,toprule=0.3mm,%
    bottomrule=0.3mm,boxsep=0.5mm,bottomrule=0.3mm,boxsep=0.5mm,%
    colback=#1!10!white,colframe=#1!50!black,#3}{#2}%
\tcbuselibrary{listings,breakable}
  \newtcbinputlisting{\Python}[2]{
    listing options={language=Python,numbers=left,numberstyle=\tiny,
      breaklines,commentstyle=\color{white!50!black}\textit},
    title=\texttt{#1},listing only,breakable,
    left=6mm,right=6mm,top=2mm,bottom=2mm,listing file={#2}}
% 三线表支持
\usepackage{booktabs}

% LaTeX logo
\usepackage{hologo}
 % 导言区
% !Mode:: "TeX:UTF-8"
% !TEX program  = xelatex
\设置信息{
    % 键 = {{中文值}, {英文值}},
    分类号 = {{}, {}},
    编号 = {{}, {}},
    UDC = {{}, {}},
    密级 = {{}, {}},
    % 仅题目(不含副标题)、系别、专业,支持手动 \\ 换行,不支持自动换行。
    题目 = {{音频分离技术中 \\ 贝斯音轨音符精准切分算法研究}, {Research on Precise Segmentation Algorithm of \\ Bass Track Notes in Audio Separation Technology}},
    % 如无需副标题,删除值内容即可,不可删除键定义。
    副标题 = {{}, {}},
    姓名 = {{卢鸿宇}, {Hongyu Lu}},
    学号 = {{12111801}, {12111801}},
    系别 = {{电子与电气工程系}, {Department of Electronic and Electrical Engineering}},
    专业 = {{信息工程}, {Information Engineering}},
    指导教师 = {{陈霏}, {Fei Chen}},
    时间 = {{2025年4月10日}, {April 10th, 2025}},
    职称 = {{教授}, {Professor}},
}
 % 论文信息
\begin{document}

\中文标题页\英文标题页

\诚信承诺书

\前序格式化
\摘要标题
% !Mode:: "TeX:UTF-8"
% !TEX program  = xelatex
\begin{中文摘要}{\LaTeX ;接口}
  如用英文写作,则英文摘要在前,中文摘要在后。

  笔者见到的毕业论文模板,大多是以文类的形式,少部分以宏包的形式,并且在模板中大多掺杂着各式各样的例子(除了维护频率高的模板),导致模板文件使用了大部分与形式格式不相关的内容,代码量巨大文档欠缺且不容易修改,出现问题需要查看宏包或者文类的源代码。于是,秉着仅提供实现最基本要求的理念,重构了之前所写的 \TeX\ 形式。由于第二年使用该模板,所以设计出的模板接口不能保证以后不发生重大变动,一切以文档为主。毕竟学校在发展初期,各类文件都在日渐完善,前几年时,学校标志及名称还发生变化,同时毕业论文的样式也发生了重大变化。但是可以保证的是,模板提供的接口均为中文形式\footnote{使用 \hologo{XeLaTeX} 特性,一方面增加辨识度,另一方面不拘泥于英文命名的规则。当然此举也有些许弊端,在此就不过多展开。},并且至少更新到 2021 年,也就是笔者毕业。模板这种东西不能保证一劳永逸,一方面学校的标准制度都在发生着改变,另一方面 \hologo{LaTeX} 的宏包也在发生着改变,早先流行的宏包可能几年后就被“淘汰”掉。因此,您的使用与反馈是我不断更新的动力,希望各位不吝赐教。
\end{中文摘要}

\begin{英文摘要}{LaTeX, Interface}
  如用英文写作,则英文摘要在前,中文摘要在后。

  \lipsum[1]
\end{英文摘要}
 % 论文摘要

\目录\cleardoublepage % 目录及换页

\正文格式化
\input{sections/examples/disclaimer.tex}
\input{sections/examples/interface.tex}
% !Mode:: "TeX:UTF-8"
% !TEX program  = xelatex

\section{一些样例}

\subsection{表格}

表格与图片可以直接通过\verbbox{\ref{<key>}}来引用,例如表\ref{table2}、图\ref{F:test-a}、图\ref{F:test-b-sub-b}。

\begin{table}[htb]
% h-here,t-top,b-bottom,优先级依次下降
    % 居中,模版已设定表格浮动体居中
    \centering
    \caption{表格的标题应该放在上方}
    \label{table}
    \begin{tabular}{lc} % 三线表不能有竖线,l-left,c-center,r-right
        \toprule
        %三线表-top 线
        Example & Result \\
        \midrule
        %三线表-middle 线
        Example1          & 0.25 \\
        Example2          & 0.36 \\
        \bottomrule
        %三线表-底线
    \end{tabular}
\end{table}

\begin{table}[htb]
    \centering
    \caption{带表注的表格的标题}
    \label{table2}
    \begin{threeparttable}
        \setlength{\tabcolsep}{0.6cm}{ % 调节表格长度
                \begin{tabular}{lc} % 三线表不能有竖线,l-left,c-center,r-right
                    \toprule
                    %三线表-top 线
                    Example & Result \\
                    \midrule
                    %三线表-middle 线
                    Example1          & 0.25\tnote{1} \\
                    Example2          & 0.36 \\
                    \bottomrule
                    %三线表-底线
                \end{tabular}
        }
        \begin{tablenotes}
            \item[1] 数据来源:南方科技大学 \LaTeX 模版 % 增加表格数据来源注释
        \end{tablenotes}
    \end{threeparttable}
\end{table}

\begin{proof}
    This is a proof written in English.
\end{proof}

\subsection{参考文献}

参考文献一般使用\verbbox{\cite{<key>}}命令,效果如是\cite{Nicholas1998Handbook},引用作者使用\verbbox{\citeauthor{<key>}},效果如是“\citeauthor{goossens1994latex}”。

\input{sections/examples/figures.tex}
\input{sections/examples/tutorial.tex}\cleardoublepage

\参考文献
  \printbibliography[heading=none]\cleardoublepage
\附录
  \input{sections/examples/appendix.tex}\cleardoublepage
\致谢
  \input{sections/examples/thanks.tex}
\end{document}
